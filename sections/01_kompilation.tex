\section{Kompilation}

\subsection{Voraussetzungen}

Es muss eine \LaTeX-Distribution wie z.B. TeX-Live (https://www.tug.org/texlive/) installiert und im PATH verfügbar sein.

Zum Kompilieren des Projekts verwenden Sie am besten \textit{latexmk}. \textit{latexmk} ist ein Skript, das die Kompilation von \LaTeX-Dateien vereinfacht. Viele Linuxdistributionen haben \textit{latexmk} bereits vorinstalliert. Ansonsten folgen Sie bitte der Installationsanleitung: https://mg.readthedocs.io/latexmk.html.

Da \textit{latexmk} in Perl geschrieben ist, muss Perl installiert sein.

\subsection{Kompilationsbefehl}

Wenn Sie alle Voraussetzungen installiert haben, führen Sie das Kommando \lstinline{latexmk -lualatex main.tex} im Root-Verzeichnis dieses Projekts aus.

\subsection{Verwendung von Visual Studio Code}

Wenn Sie Visual Studio Code zum Verfassen Ihrer Arbeit verwenden, können Sie die Extension \textit{\LaTeX-Workshop} nutzen. Die Extension nutzt standardmäßig ebenfalls \textit{latexmk}, ermöglicht Ihnen aber, das Projekt direkt aus Visual Studio Code heraus zu kompilieren.